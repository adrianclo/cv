\documentclass{article}
\usepackage[utf8]{inputenc}

\begin{document}

\section{Publications}

\begin{itemize}
    \item \textbf{Lo AC}*, Rajan N*, Telley L, Fiers M, Hilal ML, Buzzi A, Simonato M, Achsel T, \& Bagni C (manuscript in preparation). Absence of the RNA binding protein FXR2P protects against status epilepticus. * authors contributed equally
    
    \item Dom\'{i}nguez-Iturza N, \textbf{Lo AC}, Shah D, Armend\'{a}riz M, Vannelli A, Mercaldo V, Trusel M, Li KW, Gastaldo D, Santos AR, Callaerts-Vegh Z, D'Hooge R, Mameli M, Van der Linden A, Smit AB, Achsel T, \& Bagni C (2019). The autism- and schizophrenia-associated protein CYFIP1 regulates bilateral brain connectivity and behaviour. \emph{Nature Communications}. IF: 11.880
    
    \item Briz V*, Restivo L*, Pasciuto E*, Juczewski K, Mercaldo V, \textbf{Lo AC}, Baatsen P, Gunko NV, Borreca A, Girardi T, Luca R, Nys J, Poorthuis R, Mansevelder H, Fisone G, Ammassari-Teule M, Arckens L, Krieger P, Meredith R \& Bagni C (2017). The non-coding RNA BC1 regulates experience-dependent structural plasticity and learning. \emph{Nature Communications}. IF: 11.329 * authors contributed equally
    
    \item Latif-Hernandez A, Shah D, Ahmed T, \textbf{Lo AC}, Callaerts-Vegh Z, Van der Linden A, Balschun D, \& D'Hooge R (2016). Quinolinic acid injection in mouse model prefrontal cortex affects reversal learning abilities, cortical connectivity and hippocampal synaptic plasticity. \emph{Scientific Reports}. IF: 5.228
    
    \item Verreet T, Rangarajan JR, Quintens R, Verslegers M, \textbf{Lo AC}, Govaerts K, Neefs M, Leysen L, Baatout S, Maes F, Himmelreich U, D'Hooge R, \& Moons L (2016). Persistent impact of in utero irradiation on mouse brain structure and function characterized by MR imaging and behavioural analysis. \emph{Frontiers in Behavioral Neuroscience}. IF: 3.270 
    
    \item Huang Y, Skwarek-Maruszewska A, Horre K, Vandewyer E, Wolfs L, Snellinx A, Saito T, Radaelli E, Corthout N, Colombelli J, \textbf{Lo AC}, Van Aerschot L, Callaerts-Vegh Z, Trabzuni D, Bossers K, Verhaagen J, Ryten M, Munck S, D'Hooge R, Swaab DF, Hardy J, Saido TC, De Strooper B, \& Thathiah A (2015). Loss of GPR3 reduces the amyloid plaque burden and improves memory in Alzheimer's disease mouse models. \emph{Science Translational Medicine}. IF: 15.843
    
    \item Dionisio PA, Amaral JD, Ribeiro MF, \textbf{Lo AC}, D'Hooge R, \& Rodr\'{i}gues CMP (2015). Amyloid-beta pathology is attenuated by tauroursodeoxycholic acid treatment in APP/PS1 mice after disease onset. \emph{Neurobiology of Aging} \textbf{36}: 228-240. IF: 6.189
    
    \item \textbf{Lo AC}, De Maeyer JH, Vermaercke B, Callaerts-Vegh Z, Schuurkes JA, \& D'Hooge R (2014). SSP-002392, a new 5-HT4 receptor agonist, dose-dependently reverses scopolamine-induced learning and memory impairments in C57Bl/6 mice. \emph{Neuropharmacology} \textbf{85}: 178-189. IF: 4.114
    
    \item \textbf{Lo AC}, Tesseur I, Scopes DIC, Nerou E, Callaerts-Vegh Z, Vermaercke B, Treherne JM, De Strooper B, \& D'Hooge R (2013). Dose-dependent improvements in learning and memory in APPPS1-21 transgenic mice treated with the orally active A$\beta$  toxicity inhibitor SEN1500. \emph{Neuropharmacology} \textbf{75}: 458-466. IF: 4.114
    
    \item Van der Jeugd A, Vermaercke B*, Derisbourg M*, \textbf{Lo AC}*, Hamdane M, Blum D, Bu\'{e}e L, \& D'Hooge R (2013). Progressive age-related cognitive decline in the tau mice. \emph{Journal of Alzheimers Disease} \textbf{37}: 777-788. IF: 4.174 * authors contributed equally
    
    \item \textbf{Lo AC}, Iscru E, Blum D, Tesseur I, Callaerts-Vegh Z, Bu\'{e}e L, De Strooper B, Balschun D, \& D'Hooge R (2013). Amyloid and tau neuropathology differentially affect prefrontal synaptic plasticity and cognitive performance in mouse models of Alzheimer's disease. \emph{Journal of Alzheimers Disease} \textbf{37}: 109-125. IF: 4.174
    
    \item Tesseur I*, \textbf{Lo AC}*, Roberfroid A, Dietvorst S, Van Broeck B, Borgers M, Gijsen H, Moechars D, Mercken M, Kemp J, D'Hooge R, \& De Strooper B (2013). Comment on 'ApoE-directed therapeutics rapidly clear $\beta$-amyloid and reverse cognitive deficits in AD mouse models'. \emph{Science} \textbf{340}: 924-e. IF: 31.2 * authors contributed equally
    
    \item Simoes AES, Amaral JD, Nunes AF, Gomes, SE, Pereira DM, Rodrigues PM, \textbf{Lo AC}, D'Hooge R, Steer CJ, Thibodeau SN, Borralho PM, \& Rodr\'{i}gues CMP (2013). Efficient recovery of proteins from multiple source samples after trizol or trizol-LS RNA extraction and long-term storage. \emph{BMC genomics}. IF: 4.07
    
    \item Tesseur I, Pimenova AA, \textbf{Lo AC}, Ciesielska M, Lichtenthaler SF, De Maeyer JH, Schuurkes JA, D'Hooge R, \& De Strooper B (2013). Chronic 5-HT4 receptor activation decreases A$\beta$ production and deposition in hAPP/PS1 mice. \emph{Neurobiology of Aging} \textbf{34}: 1779-1789. IF: 6.189
    
    \item \textbf{Lo AC}, Callaerts-Vegh Z, Nunes AF, Rodr\'{i}gues CMP, \& D'Hooge R (2013). Tauroursodeoxycholic acid (TUDCA) supplementation prevents cognitive impairment and amyloid deposition in APP/PS1 mice. \emph{Neurobiology of Disease} \textbf{50}: 21-29. IF: 5.624
    
    \item Ramalho RM, Nunes AF, Dias RB, Amaral JD, \textbf{Lo AC}, D'Hooge R, Sebastiao AM, \& Rodr\'{i}gues CMP (2013). Tauroursodeoxycholic acid suppresses amyloid $\beta$-induced synaptic toxicity in vitro and in APP/PS1 mice. \emph{Neurobiology of Aging} \textbf{34}: 551-561. IF: 6.189
    
    \item Nunes AF, Amaral JD, \textbf{Lo AC}, Fonseca MB, Viana RJS, Callaerts-Vegh Z, D'Hooge R, \& Rodr\'{i}gues CMP (2012). TUDCA, a bile acid, attenuates amyloid precursor protein processing and amyloid-$\beta$ deposition in APP/PS1 mice. \emph{Molecular Neurobiology} \textbf{45}: 440-454. IF: 5.735
\end{itemize}

\end{document}
