% FortySecondsCV LaTeX template
% Copyright © 2019 René Wirnata <rene.wirnata@pandascience.net>
% Licensed under the 3-Clause BSD License. See LICENSE file for details.
%
% Attributions
% ------------
% * fortysecondscv is based on the twentysecondcv class by Carmine Spagnuolo 
%   (cspagnuolo@unisa.it), released under the MIT license and available under
%   https://github.com/spagnuolocarmine/TwentySecondsCurriculumVitae-LaTex
% * further attributions are indicated immediately before corresponding code


%-------------------------------------------------------------------------------
%                             ADDITIONAL PACKAGES
%-------------------------------------------------------------------------------
\documentclass[
  a4paper, 
%   showframes,
   maincolor=cvblue,
   sectioncolor=cvblue,
%   subsectioncolor=orange
%   sidebarwidth=0.4\paperwidth,
%   topbottommargin=0.03\paperheight,
%   leftrightmargin=20pt
]{fortysecondscv}

% improve word spacing and hyphenation
\usepackage{microtype}
\usepackage{ragged2e}

% take care of proper font encoding
\ifxetex
	\usepackage{fontspec}
	\defaultfontfeatures{Ligatures=TeX}
% \newfontfamily\headingfont[Path = fonts/]{segoeuib.ttf} % local font
\else
	\usepackage[utf8]{inputenc}
	\usepackage[T1]{fontenc}
% \usepackage[sfdefault]{noto} % use noto google font
\fi

% enable mathematical syntax for some symbols like \varnothing
\usepackage{amssymb}

% bubble diagram configuration
\usepackage{smartdiagram}
\smartdiagramset{
  % defaut font size is \large, so adjust to harmonize with sidebar layout
  bubble center node font = \footnotesize,
  bubble node font = \footnotesize,
  % default: 4cm/2.5cm; make minimum diameter relative to sidebar size
  bubble center node size = 0.4\sidebartextwidth,
  bubble node size = 0.25\sidebartextwidth,
  distance center/other bubbles = 1.5em,
  % set center bubble color
  bubble center node color = maincolor!70,
  % define the list of colors usable in the diagram
  set color list = {maincolor!10, maincolor!40,
  maincolor!20, maincolor!60, maincolor!35},
  % sets the opacity at which the bubbles are shown
  bubble fill opacity = 0.8,
}

\usepackage[euler]{textgreek}

%-------------------------------------------------------------------------------
%                            ADDITIONAL FUNCTIONS
%-------------------------------------------------------------------------------

\newcommand*{\cvgithub}[1]{\cvcustomdata{\faGithub}{\href{http://#1}{#1}}}

%-------------------------------------------------------------------------------
%                            PERSONAL INFORMATION
%-------------------------------------------------------------------------------
% profile picture
% \cvprofilepic{pics/profilephoto.png}
% your name
\cvname{Adrian C. \textbf{Lo}}
% job title/career
\cvjobtitle{Neuroscientist\newline Data Analyst}
% date of birth
\cvbirthday{June 30, 1984 (Belgium)}
% short address/location, use \newline if more than 1 line is required
\cvaddress{Pully (VD), Switzerland}
% phone number
\cvphone{+41 78 653 92 08}
% personal website
%\cvsite{https://pandascience.net}
% email address
\cvmail{adrianclo1984@gmail.com}
% pgp key
\cvgithub{github.com/adrianclo}
% add additional information
% \newcommand{\additional}{some more?}


%-------------------------------------------------------------------------------
%                              SIDEBAR 1st PAGE
%-------------------------------------------------------------------------------
% overwrite default icons and order of personal information
% \renewcommand{\personaltable}{%
% 	\begin{personal}[0.8em]
% 		\circleicon{\faKey}      & \cvkey  \\
% 		\circleicon{\faAt}       & \cvmail \\
% 		\circleicon{\faGlobe}    & \cvsite \\
% 		\circleicon{\faPhone}    & \cvphone \\
% 		\circleicon{\faEnvelope} & \cvaddress \\
% 		\circleicon{\faInfo}     & \cvbirthday \\
% 		% add another line
% 		\circleicon{\faQuestion} & \additional
% 	\end{personal}
% }

% add more profile sections to sidebar on first page
\addtofrontsidebar{
	% include gosquare national flags from https://github.com/gosquared/flags;
	% naming according to ISO 3166-1 alpha-2 country codes
	\graphicspath{{pics/flags/}}
    
    \profilesection{About Me}
	\aboutme{
		I have a background in theoretical psychology and \textbf{statistics}. During the last 5 years I studied and analyzed rodent behavior and molecular biology, but also gained expertise in developing \textbf{R programs}, \textbf{shiny apps} and \textbf{automated reports}. With these tools, I improved the speed and efficiency of data-processing for myself as well as colleagues.
	}
	
	\profilesection{Languages}
		\pointskill{\flag{BE.png}}{Dutch (native)}{5}
		\pointskill{\flag{GB.png}}{English}{5}
  	    \pointskill{\flag{FR.png}}{French}{4}
  	    \pointskill{\flag{CN.png}}{Chinese (Cantonese)}{4}
  	    \pointskill{\flag{DE.png}}{German}{2}
  	    
	\profilesection{Computer Skills}
	
		\pointskill{}{R}{5}
		\pointskill{}{R Markdown}{5}
		\pointskill{}{Visualization (ggplot2)}{5}
		\pointskill{}{Excel}{5}
		\pointskill{}{Excel (macro and VBA)}{4}
		\pointskill{}{Tableau (BI)}{4}
		\pointskill{}{Machine Learning}{4}
		\pointskill{}{R Shiny}{4}
		\pointskill{}{Git/Github}{3}
		\pointskill{}{SQL}{3}
		\pointskill{}{Python}{3}
		%\pointskill{}{SPSS}{3}
  	    \pointskill{}{HTML}{2}
  	    \pointskill{}{\LaTeX}{2}
  	    \pointskill{}{SAS}{1}
}

%-------------------------------------------------------------------------------
%                              SIDEBAR 2nd PAGE
%-------------------------------------------------------------------------------
\addtobacksidebar{
	\profilesection{Soft Skills}
	\begin{figure}\centering
		\smartdiagram[bubble diagram]{
			\textcolor{white}{\textbf{Project}} \\ 
			\textcolor{white}{\textbf{Management}}, % center bubble	
			\textcolor{black!90}{Problem Solving},
			\textcolor{black!90}{Planning},
			\textcolor{black!90}{Risk Management},
			\textcolor{black!90}{Communication},
			\textcolor{black!90}{Coaching}
		}
	\end{figure}

    \begin{figure}\centering
		\smartdiagram[bubble diagram]{
			\textcolor{white}{\textbf{Personality}}, % center bubble	
			\textcolor{black!90}{Proactive},
			\textcolor{black!90}{Exploratory},
			\textcolor{black!90}{Analytical Mind},
			\textcolor{black!90}{Team Player}
		}
	\end{figure}
\newline \newline

	\profilesection{Extra-Curricular \newline Activities}
    \wheelchart{4em}{2em}{
  	45/3em/maincolor!50/Sports,
  	10/3em/maincolor!15/Travel,
  	25/4em/maincolor!40/Music,
  	20/3em/maincolor!20/Nature
	}
	
	\vspace*{0.5cm}
	\chartlabel{\faCar \hspace{0.1cm} Driver's license: B (2003)} 
}

%-------------------------------------------------------------------------------
%                         TABLE ENTRIES RIGHT COLUMN
%-------------------------------------------------------------------------------
\begin{document}

\makefrontsidebar

\cvsection{Work Experience}
\begin{cvtable}
	\cvitem{2021 -- present}{Clinical Data Manager}{MSD, Switzerland}{
	Review and curation of clinical trial data}
	\cvitem{2016 -- 2021}{Neuroscientist}{Université de Lausanne, Switzerland}{
	- Post-doctoral research on the role of RNA binding protein FXR2P in status epilepticus: Behavioral and molecular evaluation (Laboratory of Prof. Claudia Bagni)\newline
	- Reference person within the research group on issues related to statistics and programming\newline
	- Responsible for the organisation of the departmental stockroom}
	\cvitem{2014 -- 2015}{Neuroscientist}{KU Leuven, Belgium}{
		Post-doctoral research on cue competition and contextual fear learning in rodents and humans. (Laboratory of Prof. Bram Vervliet)}
\end{cvtable}

\cvsection{Education}
\begin{cvtable}
    %\cvitem{2015}{Summer school}{Center for Excellence, Belgium}{Emotional Learning and Memory in Health and Psychopathology}
    %\cvitem{2014}{\textbf{Neuroscience Summer Program}}{RIKEN BSI, Japan}{}
    %\cvitem{2011}{\textbf{Training in the use of the \href{https://www.tse-systems.com/product-details/intellicage}{\color{blue}{IntelliCage}}}}{University of Zürich, Switzerland}{}
	\cvitem{2008 -- 2013}{PhD in Psychology}{KU Leuven, Belgium}{}
	%\cvitem{2008 -- 2013}{PhD student, Neuroscientist}{KU Leuven, Belgium}{Dissertation: "Mouse models of Alzheimer's disease: behavioral validity and use in preclinical therapy evaluation". Mentors: Prof. Rudi D'Hooge and Prof. Bart De Strooper}
	%\cvitem{2007 -- 2008}{Intership}{KU Leuven, Belgium}{Dissertation: "Visual spatial learning in C57BL/6J mice: proximal and distal cue-related learning in the Barnes maze and radial arm maze. The involvement of dorsal hippocampus in the Barnes maze". Mentors: Prof. Rudi D'Hooge and Dr. Ilse Gantois}
	\cvitem{2003 -- 2008}{Master of Science in Theoretical Psychology}{KU Leuven, Belgium}{}
	%\cvitem{2003 -- 2008}{Master of Science in Theoretical Psychology}{KU Leuven, Belgium}{Dissertation: "The relationship between working memory and (non-)focal search processes in prospective memory". Mentor: Prof. Géry Van Outryve d'Ydewalle}
\end{cvtable}
    
\cvsection{Certificates and Courses}
    \cvitem{02/2021}{Analyzing Data in Tableau}{Datacamp}{}
    \cvitem{12/2020}{Databases and SQL for Data Science}{IBM, Coursera}{}
    \cvitem{12/2019}{Advanced R Shiny}{SIB, Switzerland}{}
    \cvitem{01/2019}{Data Management Plan}{SIB, Switzerland}{}
    \cvitem{10/2018}{Project Management}{EPFL, Switzerland}{}
    \cvitem{09/2018}{Introduction to Data Analysis with Python}{EPFL Extension School, Switzerland}{}
    \cvitem{06/2018}{Statistical Methods for Big Data in Life Sciences and Health with R}{SIB, Switzerland}{}
    \cvitem{09/2015}{Introduction to SAS}{LSTAT, Belgium}{}
    \cvitem{05/2015}{Text Mining with R}{KU Leuven, Belgium}{}
    \cvitem{09/2013}{FELASA C - Laboratory Animal Sciences}{KU Leuven, Belgium}{}

\cvsection{My R programs portfolio}
\textbf{meaR} (public repository: click  \href{https://github.com/adrianclo/meaR}{\color{blue}{here}} to review it) \newline
The text files from \href{https://www.multichannelsystems.com/products/vitro-mea-systems}{\color{blue}{Micro-Electrode Arrays}} contain \emph{in vitro} electrophysiological measurements interspersed with text.  The numeric \textbf{data are extracted} from the text file and a master datafile is assembled. meaR then performs calculations for a variety of electrophysiological parameters and visualizes spike and burst activity for all 60 electrodes over time \newline

\textbf{phenotyper} (private repository, available for discussion) \newline
For the processing and analysis of \href{https://www.sylics.com/bioinformatics/cognitionwall/}{\color{blue}{Phenotyper}} data, we can use a cloud service upon payment. Through \textbf{reverse engineering}, I designed the phenotyper program that performs similarly to the cloud service and calculates additional behavioral parameters \newline

\textbf{easyGeno} (private repository, available for discussion) \newline
Mouse genotyping is a tedious process that requires several steps prior to the wet lab work:  identification of the sample's model, pre-mix calculations, and planning of the assembly plates for PCR and electrophoresis. These can easily take up to half a day time. With easyGeno, an \textbf{automated report} is created with R Markdown that contains all these steps ready for the user to follow and optimized for the \href{https://www.qiagen.com/us/service-and-support/learning-hub/technologies-and-research-topics/sample-quality-control/instruments/qiaxcel-advanced/}{\color{blue}{QIAxcel apparatus}}. Finally, I developed a follow-up module that extracts the result from the QIAxcel pdf report and \textbf{cross-references with our database file} to automate band identification \newline

\textbf{unidamr} (private repository, available for discussion) \newline
Through an \textbf{interactive Shiny application}, behavioral data from \emph{Drosophila} are analyzed, categorized as either sleep or awake state, and several parameters are calculated and analyzed

\newpage
\makebacksidebar

\cvsection{Teaching Experience}
    \cvitem{2019-2020}{Coding Club}{Université de Lausanne, Switzerland}{Interactive course between PhD students and Postdocs on how to use R for data import, manipulation, visualization and analysis}
    \cvitem{09/2015}{Workshop at Summer School}{KU Leuven, Belgium}{Subject: "The use of rodent models in fear conditioning, learning and memory}
    \cvitem{2013}{Bachelor Course at KU Leuven}{B-KUL-P0M20B}{How to use SPSS for basic data manipulation, statistics and SPSS output interpretation}

\cvsection{Conferences and Presentations}
    \cvitem{2018}{NCCR-SYNAPSY Conference}{Geneva, Switzerland}{Cognitive flexibility in a mouse model for Fragile X Syndrome}
    \cvitem{2014}{RIKEN Brain Science Institute}{Tokyo, Japan}{Treatment with tauroursodeoxycholic acid modulates \textgamma-secretase activity and rescues memory deficits in APP/PS1 mice, an AD mouse model}
    \cvitem{2012}{International Stockholm/Springfield symposium on advances in Alzheimer's disease}{Stockholm, Sweden}{Behavioural effects of selenium in mouse models of Alzheimer’s disease}
    \cvitem{2010}{Forum of European Neurosciences}{Amsterdam, The Netherlands}{Reversible changes in neurocognitive performance and hippocampal synaptic plasticity in tau mutant mouse lines}
    
\cvsection{Publications (6 most recent)}
For the full list, \href{https://www.researchgate.net/profile/Adrian_Lo/research}{\color{blue}{please click here}} \newline

\begin{cvtable}
    \cvitem{2021}{\emph{BioRxiv}}{}{\textbf{Scopolamine blocks context-dependent reinstatement of fear responses in rats} \href{https://www.biorxiv.org/content/10.1101/2021.02.24.432279v1}{\color{blue}{[doi]}} \newline Vercammen, LM, \textbf{Lo AC}, D'Hooge R, Vervliet B.}
	\cvitem{}{\emph{EMBO Reports}}{}{\textbf{Absence of RNA binding protein FXR2P prevents prolonged phase of kainate-induced seizures} \href{https://www.embopress.org/doi/full/10.15252/embr.202051404}{\color{blue}{[doi]}} \newline \textbf{Lo AC}, Rajan N, Gastaldo D, Telley T, Hilal ML, Buzzi A, Simonato M, Achsel T, Bagni C.}
	\cvitem{2019}{\emph{Nature Communications}}{}{\textbf{The autism- and schizophrenia-associated protein CYFIP1 regulates bilateral brain connectivity and behaviour} \href{https://www.nature.com/articles/s41467-019-11203-y}{\color{blue}{[doi]}} \newline Domínguez-Iturza N, \textbf{Lo AC}, Shah D, Armendáriz M, Vannelli A, Mercaldo V, Trusel M, Li KW, Gastaldo D, Santos AR, Callaerts-Vegh Z, D'Hooge R, Mameli M, Van der Linden A, Smit AB, Achsel T, Bagni C. }
	\cvitem{2017}{\emph{Nature Communications}}{}{\textbf{The non-coding RNA BC1 regulates experience-dependent structural plasticity and learning} \href{https://www.nature.com/articles/s41467-017-00311-2}{\color{blue}{[doi]}} \newline Briz V, Restivo L, Pasciuto E, Juczewski K, Mercaldo V, \textbf{Lo AC}, Baatsen P, Gounko NV, Borreca A, Girardi T, Luca R, Nys J, Poorthuis RB, Mansvelder HD, Fisone G, Ammassari-Teule M, Arckens L, Krieger P, Meredith R, Bagni C.}
	\cvitem{2014}{\emph{Neuropharmacology}}{}{\textbf{SSP-002392, a new 5-HT4 receptor agonist, dose-dependently reverses scopolamine-induced learning and memory impairments in C57Bl/6 mice}  \href{https://www.sciencedirect.com/science/article/abs/pii/S0028390814001804?via\%3Dihub}{\color{blue}{[doi]}} \newline \textbf{Lo AC}, De Maeyer JH, Vermaercke B, Callaerts-Vegh Z, Schuurkes JA, D'Hooge R.
	}
	%\cvitem{2013}{\emph{Neuropharmacology}}{}{\textbf{Dose-dependent improvements in learning and memory deficits in APPPS1-21 transgenic mice treated with the orally active A\textbeta toxicity inhibitor SEN1500}  \href{https://www.sciencedirect.com/science/article/abs/pii/S0028390813003973?via\%3Dihub}{\color{blue}{[doi]}} \newline \textbf{Lo AC}, Tesseur I, Scopes DI, Nerou E, Callaerts-Vegh Z, Vermaercke B, Treherne JM, De Strooper B, D'Hooge R.
	%}
	\cvitem{2013}{\emph{Science}}{}{\textbf{Comment on "ApoE-directed therapeutics rapidly clear \textbeta-amyloid and reverse deficits in AD mouse models"} \href{https://science.sciencemag.org/content/340/6135/924.5.long}{\color{blue}{[doi]}} \newline Tesseur I*, \textbf{Lo AC}*, Roberfroid A, Dietvorst S, Van Broeck B, Borgers M, Gijsen H, Moechars D, Mercken M, Kemp J, D'Hooge R, De Strooper B. * authors contributed equally
	}
\end{cvtable}

\cvsignature

\end{document} 
